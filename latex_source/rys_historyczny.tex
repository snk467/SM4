\section{Rys historyczny}

Algorytm „SMS4” został wymyślony przez Shu-Wang Lu. Po raz pierwszy został opublikowany w 2003 r. jako część 	"Information technology -- Telecommunications and information exchange between systems -- Local and metropolitan area networks -- Specific requirements -- Part 11: Wireless LAN Medium Access Control (MAC) and Physical Layer (PHY) Specifications", a następnie opublikowany niezależnie w 2006 r. przez SCA(State Cryptography Administration of China) (wówczas OSCCA - Office of State Commercial Cryptography Administration) jako SMS4 - Cryptographic Algorithm For Wireless LAN Products. Został opublikowany jako branżowy standard kryptograficzny i przemianowany na „SM4” w 2012 r. przez SCA, a ostatecznie sformalizowany w 2016 roku jako Chiński Standard Narodowy - "Information security technology -- SM4 block cipher algorithm".\\

SM4 został pierwotnie stworzony do użytku w ochronie sieci bezprzewodowych i jest zgodny z chińskim National Standard for Wireless LAN WAPI (Wired Authentication and Privacy Infrastructure), czyli alternatywą dla mechanizmów bezpieczeństwa określonych w IEEE 802.11i.
Został przedłożony Międzynarodowej Organizacji Normalizacyjnej ISO przez Chińskie Stowarzyszenie Normalizacyjne SAC. \\

Zarówno WAPI, jak i IEEE 802.11i zostały zaproponowane jako poprawki bezpieczeństwa do normy ISO/IEC 8802-11.
Oba schematy wykorzystują dwa różne szyfry blokowe do szyfrowania danych:
- WAPI wykorzystuje szyfr blokowy SMS4, podczas gdy IEEE 802.11i używa szyfru blokowego AES. \\

W marcu 2006 roku IEEE 802.11i został zatwierdzony jako ISO/IEC 8802-11 WLAN, natomiast WAPI został częściowo odrzucony, ze względu na nieujawniomy szyfr SMS4. Jednakże, ponieważ WAPI jest nadal oficjalnie wymagany dla chińskiego standardu krajowego,
jest nadal używany w chińskiej branży WLAN i wielu międzynarodowych
korporacjach, takich jak SONY, które wspierają WAPI w odpowiednich produktach. \\



Najnowszy standard SM4 został zaproponowany przez SCA (wówczas OSCCA), znormalizowany przez TC 260 Administracji Normalizacyjnej Chińskiej Republiki Ludowej (SAC) i został opracowany przez następujące osoby w Centrum Badań Danych i Bezpieczeństwa Komunikacji (Centrum DAS) Chińskiej Akademii Nauk, Chińskie Centrum Testowania Kryptografii Komercyjnej oraz Pekińska Akademia Informatyki i Technologii (BAIST):

\begin{itemize}
    \item Shu-Wang Lu
    \item Dai-Wai Li
    \item Kai-Yong Deng
    \item Chao Zhang
    \item Peng Luo
    \item Zhong Zhang 
    \item Fang Dong
    \item Ying-Ying Mao
    \item Zhen-Hua Liu
\end{itemize}


SM4 został również ostatecznie znormalizowany w ISO/IEC 8802-11 przez Międzynarodową Organizację Normalizacyjną w 2017 r.














